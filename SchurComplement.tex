
\chapter{Property of Schur Complement}
\section*{Overview}


\marginpar{
  \begin{marginnotes}
    Main result in this part comes from the book ``The Schur Complement and its applications''
  \end{marginnotes}
}

In this part we give some basic result for the Schur Complement.

Consider a matrix $M$,
$$
M=\left(\begin{array}{cc}
  P & Q \\
  R & S
  \end{array}\right)
$$
,
we define the 
\Kchange{Schur Complement of $P$ in the matrix $M$}
 as 
\begin{align}
  (M/P) = S - RP^{-1 }Q
\end{align}

\begin{lemma}
  Schur determinant formula


  P,Q,S,R all $n \times n$ matrix, then we have

  $\det(M) = \det(PS - RQ)$

  and

  $\det M = \det(P) \det(S-RP^{-1}Q)$
\end{lemma}

\begin{definition}
  The \Kchange{inertia} of hermitian matrix $H$ is the triple tuple
  $In(H)= \{\pi,\nu,\delta\}$ is the number of positive, negative and zero eigenvalue.
\end{definition}

and we have the result
$$
H=\left(\begin{array}{cc}
  H_{11} & H_{12} \\
  H_{12}^* & H_{22}
  \end{array}\right)
$$

then

$In(H) = In(H_{11}) + In(H/H_{11})$

For matrix $M$,
$$
M=\left(\begin{array}{cc}
  A & B \\
  C & D
  \end{array}\right)
$$

we have 
$$M/D = A - BD^{-1}C$$
and
$$M/A = D - BA^{-1}C$$

\begin{lemma}
  Schur formula
  if $M$ square, and $A$ nonsingular.
  $\det(M/A) = \det(M)/ \det(A)$
\end{lemma}

For a Matrix $A$ of size $n \times n$,
we define the index set $\alpha$ and $\beta$ are subsets of $\{1,\dots,n\}$.

Then we define $A[\alpha,\beta]$ is the submatrix with row index $\alpha$ and column index $\beta$.
And we say $A[\alpha] = A[\alpha,\alpha]$.
Then $A/A[\alpha,\beta]$ is the schur complement of $A[\alpha,\beta]$ in the matrix $A$.

$$
A/A[\alpha,\beta] = A[\alpha^c,\beta^c] - A[\alpha^c,\beta](A[\alpha,\beta]^{-1})A[\alpha,\beta^c]
$$

And we denote $A/A[\alpha]$ as $A/\alpha$.


For matrix $A$,
$$
A=\left(\begin{array}{cc}
  A_{11} & A_{12} \\
  A_{12}^* & A_{22}
  \end{array}\right)
$$

$A >0$ if and only if $A_{11}>0$ and $A/A_{11} >0$

$A \geq 0$ if and only if $A_{11}>0$ and $A/A_{11} \geq 0$

if $A \geq 0$
and $A_{11} >0$,
then $A/A_{11} = A_{22} - A_{12}A_{11}^{-1} a_{12} \geq 0$
so
$A_{22} \geq A/A_{11} \geq 0$,
$\det(A_{22})\geq 0$

\begin{lemma}
  A,B are $n\times n$ positive matrices, then 
  $\det(A+B) \geq \det(A) + \det(B)$
\end{lemma}

\section*{Eigenvalue and singular value of schur complement}

$\mH_n$ is $n\times n$ Hermitian matrix sets.

For $A \in \mH_n$ we define eigenvalues 
$\lambda_1(A) \geq \lambda_2(A) \geq \dots,\geq \lambda_n(A)$.


For $A \in \mC^{m\times n}$ we define singular value 
$\sigma_1(A) \geq \sigma_2(A) \geq \dots,\geq \sigma_n(A)$.




\begin{lemma}
  Cauthy eigenvalue interlacing theorem

  for 
  $$
  H=\left(\begin{array}{cc}
  A & B \\
  B^* & D
  \end{array}\right)
$$
where $A$ is $r \times r$ and $H$ is $n \times n$.

we have
$$
\lambda_i(H) \geq \lambda_i(A)\geq \lambda_{i+n-r}(H)
$$
\end{lemma}

\begin{lemma}
  $H \in \mH_n$, $\alpha$ is a index set of size $k$, $1\leq k<n$,
  if $H[\alpha]$ positive definite,
  then 
  $$
\lambda_i(H) \geq \lambda_i(H/\alpha \bigoplus \bzero)  \geq \lambda_{i+k}(H)
$$
\end{lemma}

\begin{corollary}
  $H$ is a $n \times n$ positive semidefinite matrix.
  $H[\alpha]$ is $k \times k$.
  then 
  $$
\lambda_i(H) \geq \lambda_i(H/\alpha )  \geq \lambda_{i+k}(H)
$$

$$
\lambda_i(H) \geq \lambda_i(H[\alpha^c] )  \geq \lambda_{i+k}(H/\alpha) \geq \lambda_{i+k}(H)
$$
\end{corollary}


\begin{corollary}
  $H$ is a $n \times n$ positive semidefinite matrix.
  $\alpha$ and $\alpha^\prime$ are to nonnull index set.
  $\alpha^\prime \subset \alpha \subset \{1,2,\dots,n\}$

  if $H[\alpha]$ non-singular,
  for every $i  =1,2,\dots,n - |\alpha|$

  $$
  \lambda_i(H/\alpha^\prime)
  \geq 
  \lambda_i(H[\alpha^\prime\cap \alpha^c]/\alpha^\prime)
  \geq
  \lambda_i(H/\alpha)
  \geq 
  \lambda_{i+|\alpha| - |\alpha^\prime|}(H/\alpha^\prime)
  $$
\end{corollary}
